\documentclass[12pt]{article}

\usepackage{biblatex}
\addbibresource{references.bib}

\usepackage{graphicx} % Required for inserting images
\usepackage[a4paper, margin=1in]{geometry}
\usepackage{float}
\usepackage{multirow}
\usepackage{adjustbox}


\title{Procedural world generation and the challenges of rendering in non-Euclidean spaces}
\author{Aleksy Bałaziński \and Karol Denst}
\date{\today}

\begin{document}

\maketitle
\newpage

\tableofcontents
\newpage

\section{Abstract} % A.B.
Procedural generation, as a method used in creating video games, has experienced a significant increase in popularity in recent years.
It has been employed extensively in acclaimed titles such as \textit{Minecraft} and \textit{No Man's Sky} to create virtually infinite worlds that the player is free to interact with.
On the other hand, a recent game \textit{Hyperbolica} shed a light onto a novel idea of making the virtual worlds even more interesting by setting them in non-Euclidean spaces.
The objective of this thesis is to create a small video game incorporating both of the aforementioned concepts.


\subsection{History of changes}
\begin{table}[H]
\centering
\begin{tabular}{lllll}
\hline
Date       & Author                                                                  & Description    & Version &  \\ \hline
18.10.2023 & \begin{tabular}[c]{@{}l@{}}Aleksy Bałaziński\\ Karol Denst\end{tabular} & Initial description of the project & 1.0     &  \\
           &                                                                         &                &         &  \\
           &                                                                         &                &         &  \\ \hline
\end{tabular}
\end{table}
 
\section{Vocabulary} % A.B.
\textbf{Hyperbolic geometry} -- geometry obtained by replacing Euclid's parallel postulate with the following axiom: "For any given line $R$ and point $P$ not on $R$, in the plane containing both line $R$ and point $P$ there are at least two distinct lines through $P$ that do not intersect $R$." \\
\textbf{Elliptic geometry} -- geometry obtained by replacing Euclid's parallel postulate with the following axiom: "Through any point in the plane, there exist no lines parallel to a given line."\\
\textbf{Procedural world generation} -- a technique that allows for creating content (commonly terrain) in a video game, animated movie etc. in an automated manner, that is by the means of an algorithm rather than using manually crafted assets.\\
\textbf{NPC} -- non-player character.\\
\textbf{FPS} -- frames per second.\\
\textbf{GB} -- gigabyte.\\
\textbf{RAM} -- random access memory.\\


\section{Specification} % K.D.

\subsection{Executive summary}

This application is supposed to combine two effects often seen in video games. The first effect is world generation known from games such as Minecraft. The second one is Moving the gameplay to non-Euclidean spaces. This effect is know form games like Hyperbolica. The application will be available on Windows and will make use of OpenGL for rendering.

\subsection{Functional requirements}

\begin{table}[H]
    \begin{tabular}{p{0.45\linewidth}|p{0.45\linewidth}|p{0.07\linewidth}}
    \hline
        \textbf{As a player I want to} & \textbf{So that} & \textbf{Flags} \\ \hline
        Start a game in Euclidean space & I can explore euclidean spaces & Must \\ \hline
        Start a game in hyperbolic space & I can explore hyperbolic spaces & Must \\ \hline
        Start a game in spherical space & I can explore spherical space & Must \\ \hline
        Have gravity in the game & I have Earth-like experience & Must \\ \hline
        Can collide with other objects in the game & I can interact with other objects & Must \\ \hline
        Be able to move around & I can explore the world & Must \\ \hline
        Be able to jump & I can jump & Must \\ \hline
        Be able to shoot projectiles & I can see how things move in non-Euclidean spaces & Must \\ \hline
        Have a limited number of projectiles in the inventory & The game is more realistic & Must \\ \hline
        Be able to interact with NPCs & So that there is more to do in the game & Should \\ \hline
        Have NPCs move around & The world looks more rich & Must \\ \hline
        Edit the terrain & I am able to view any structure in different spaces & Must \\ \hline
        See the marker of the terrain editor & I can decide where the terrain modification is going to take place & Must\\ \hline
        Have different terrain generated every time I load a game & I can explore different terrains & Must \\ \hline
        Have light sources that illuminate other objects & I can see what lighting looks like in different spaces & Must \\ \hline
        See a night/day cycle & I can compare the world during the day and the night & Should \\ \hline
        Have a vehicle I can enter & Move around faster & Must \\ \hline
        Have a vehicle with suspension springs & I can drive around comfortably & Must \\ \hline
        Have a vehicle that has two top velocities & I can drive faster if I get bored & Must\\ \hline
        
        Have models for the player and NPCs & Characters look good & Must \\ \hline
        Have model for the vehicle & Vehicle looks good & Must\\ \hline
        Have an inventory & I can manage different items & Must \\ \hline
        Have objects cast shadows & I can explore what shadows look like in different spaces & Should \\ \hline
        Have a way to save/load a game & Be able to store interesting worlds & Must \\ \hline
        Be able to save a game & I can save interesting worlds & Must \\ \hline
        Be able to load a game & I can revisit interesting worlds & Must \\ \hline
        Have a CLI & I can manage my saves and other game options & Must \\ \hline
        Have a \texttt{help} option in the CLI & I know what commands are available and what they do & Must \\ \hline
        Have GUI & I can manage my saves easily & Should \\ \hline
        
    \end{tabular}
\caption{Player functional requirements}
\end{table}

\subsection{Non-functional requirements}

\begin{table}[H]
    \begin{tabular}{|l|p{2.2cm}|p{11cm}|}
    \hline
        \textbf{Requirements area} & \textbf{Requirement Number} & \textbf{Description} \\ \hline
        Functionality & 1 & The application should offer world generation in 3 different spaces as well as means to explore them. \\ \hline
        Usability & 2 & The application should have a menu and gameplay controls. \\ \hline
        Reliability & 3 & The application should run without crashing or memory leaks for extended periods of time. \\ \hline
        Performance & 4 & The application should run at 60 FPS with a resolution of 1920x1080 on a computer with at least Intel Core i7-1260P and 16 GB of RAM. \\ \hline
        Sustainability & 5 & The game should save logs and world data to make solving any potential bugs easier. \\ \hline
    \end{tabular}
\caption{Project non-functional requirements}
\end{table}


\section{Project schedule} % K.D.

\begin{table}[H]
    \begin{tabular}{lll}
    \hline
        \textbf{Task} & \textbf{Begin Date} & \textbf{End Date} \\ \hline
        Implement a basic OpenTK application & \multirow{11}{*}{01/10/2023} & \multirow{11}{*}{19/10/2023} \\
        Implement terrain generation         &                              &  \\
        Implement lighting                   &                              &  \\
        Implement mining/building            &                              &  \\
        Implement collision physics          &                              &  \\
        Add gravity                          &                              &  \\       
        Add a drivable car                   &                              &  \\
        Implement loading and displaying models &                           &  \\
        Implement model animation &                                         &  \\
        Implement shooting projectiles &                                    &  \\ 
        Implement ray casting &                                             &  \\
        Add NPCs &                                                          &  \\
        Implement inventory system &                                        &  \\
        Implement a simple CLI                                              &  \\
        Implement load and save system for the game                         & \\ \hline
        Add better models for NPCs, vehicles and the player & 19/10/2023 & 02/11/2023 \\ \hline
        Add shadows & 19/10/2023 & 09/11/2023 \\ \hline
        Add day/night cycle & 02/11/2023 & 16/11/2023 \\ \hline
        Add a menu & 17/11/2023 & 01/12/2023 \\ \hline
        Implement hyperbolic space & \multirow{3}{*}{01/10/2023} & \multirow{3}{*}{07/12/2023} \\
        Implement spherical space &  &  \\ 
        Add different terrains to the generation &  & \\ \hline
    \end{tabular}
\caption{Project schedule}
\end{table}

\section{Risk analysis} % A.B.

\begin{table}[H]
\centering
\begin{tabular}{|p{0.15\linewidth}|p{0.4\linewidth}|p{0.4\linewidth}|}
\hline
\textbf{SWOT} & \textbf{Threats} & \textbf{Opportunities} \\ \hline
\textbf{Internal} 
& \begin{itemize}
    \itemsep0em 
    \item Not enough time to implement all features
    \item Bad design decisions hindering further development
    \item Failure to implement rendering in non-Euclidean spaces
    \end{itemize}
&
\begin{itemize}
\itemsep0em
    \item Project developed with best practices in mind, resulting in easily extensible code.
    \item Project employs the techniques of object oriented programming to make it easier to manage state by exposing simple, well-defined interfaces.
    \item Project is well documented, making it easier to understand the code base.
\end{itemize}
\\ \hline
\textbf{External} 
& 
\begin{itemize}
    \itemsep0em
    \item External libraries don't support highly specific features
    \item Poor documentation (or lack of thereof) for external libraries
    \end{itemize}
& \begin{itemize} 
    \itemsep0em 
    \item Sparking interest in non-Euclidean geometries 
    \item Proposed methods of rendering non-Euclidean geometries could be used in other projects
  \end{itemize}\\  \hline
\end{tabular} 

\end{table}

\newpage
\section{References}
\printbibliography

\end{document}
