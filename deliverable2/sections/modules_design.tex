\section{Modules design and system architecture}
% interfaces
% dependencies
% layers


The \textit{Hyper} video game is a relatively big project (it comprises around 160 classes), so naturally we have to contain the discussion to only the most important parts of the system.
Conceptually, there are four different "areas" spanned by the project:
\begin{enumerate}
    \item game objects management,
    \item procedural world generation,
    \item rendering,
    \item user interface.
\end{enumerate}

In what follows, we will discuss these areas in more detail.

\subsection{Game objects management}
A game object, such as a humanoid-looking bot or a car most often can be understood as existing in three different planes simultaneously:
\begin{enumerate}
    \item the visual layer,
    \item the physical layer,
    \item the logical layer.
\end{enumerate}
The visual layer of an object describes how the object looks like during the gameplay.
This side of a game object we will call a \textit{model}.
The information about models (vertices, normal vectors, UV mappings, textures, and animations) is read from COLLADA and PNG files and stored in the \texttt{Model} class.
It's important to note that if multiple game objects look the same, the model information is shared between them in the form of a single \texttt{ModelResource} class instance.
More specifically, the \texttt{ModelResource} class is abstract, and its derived classes are singletons.
Animated models additionally use the \texttt{Aminator} class which provides a way to calculate the bone transforms based on the elapsed time.
Some models are \textit{compound}.
The car model, for example, stores information about the look of the wheels and body separately.

The physical layer defines the physical properties of the game object at any given moment.
Some basic ones include inertia, angular and linear velocity, and friction;
in the case of more complex objects such as a car, the description contains information about \textit{constraints}.
The constraints define the relative positions of different elements of the object and their velocities.
It's worth noting that the position constraints don't have to be "stiff" (they are modeled as springs).
Technically, once a game object is added to the simulation, it is represented by one (or more in the case of compound bodies) \textit{body handle}.
Therefore, each game object has to implement the \texttt{ISimulationMember} interface which has a property that returns the list of all body handles associated with the given game object.

The logical layer is the soul of each object.
NPCs move in a way given by a simple algorithm;
the player can inflict damage to NPCs by shooting projectiles, and so on.
Game objects can react to collisions with other objects by registering for contact callbacks, i.e. implementing the \texttt{IContactEventListener} interface.
They can also move (either on their own, like NPCs, or due to the player input) by registering input callbacks, i.e. implementing the \texttt{IInputSubscriber} interface.

\subsection{Procedural world generation}
This part of the system is described in more detail in \ref{chunk_generator}.

\subsection{Rendering}
Every game object exposes a \texttt{Render()} method.
The \texttt{Render} methods' signatures differ slightly across different game objects, but they usually take camera position, shader and curve as arguments.
At the time of writing, there are three shaders available for 3D rendering:
\begin{enumerate}
    \item light source shader,
    \item model shader,
    \item object shader.
\end{enumerate}
Model shader is used for rendering animated models, whereas light source and object shaders are used for rendering "static" bodies.
The light source shader is extremely basic: it doesn't take into account other light sources and colors the body uniformly.
The \texttt{Render} method typically interacts directly with the OpenGL interface, i.e. sets up the uniforms, binds VAOs and makes a draw call.
In the case of 2D rendering, \texttt{HudShader} class is used.

\subsection{User interface}
The inner workings of the UI are described in \ref{sprite_renderer}.
All UI elements (tip texts, icons, etc.) implement the \texttt{IHudElement} interface which has \texttt{Visible} property and \texttt{Render()} method.
The \texttt{Visible} property determines if the UI element will be shown or not.
The exact way that happens is implementation-dependent.
For example, a tip text "press C to enter the car" will be shown if the player is close enough to the vehicle.