\subsection{Executive summary}

The video game can be started in one of three modes:
\begin{enumerate}
    \item Euclidean geometry
    \item spherical geometry,
    \item hyperbolic geometry.
\end{enumerate}
The last two modes will allow visualizing what living in a 3D non-Euclidean space would look like.
In the video game, the player will be able to explore a procedurally generated world and interact with it by modifying the terrain (building new structures, digging tunnels, etc.).
In the case of Euclidean and hyperbolic geometry modes, the world will be infinite and generated on the fly.
% The 'fight against' stuff in the next sentence wasn't discussed yet, so feel free to give your opinion. But I think we could do this.
The world will also be inhabited by other creatures that the player will have to fight against by, for example, shooting a gun.
Showing the motion of the projectiles will also provide an opportunity to show that in non-Euclidean geometries they appear to move along curved paths.
To make exploring a vast environment offered in modes 1. and 3. easier, the player will find a vehicle that they can get into and drive around in.




